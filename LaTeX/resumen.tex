\chapter*{Resumen}
\addcontentsline{toc}{chapter}{Resumen. Palabras clave}

Este trabajo supone la simulación y el análisis de un modelo matemático que describe la respuesta termoluminiscente (TL) del LiF:Mg,Ti; un material ampliamente utilizado en la dosimetría de radiación. Se han desarrollado dos modelos diferentes para describir estos procesos físicos mediante la resolución de un sistema de ecuaciones diferenciales: uno con un factor de frecuencia constante y dependiente de cada trampa y otro con un factor de frecuencia que depende de la temperatura, fundamentado en los principios de la física estadística cuántica.

\vspace{10pt}

Para este Trabajo Fin de Grado se han simulado una serie de curvas termoluminiscentes en un entorno de Python con el que se ha resuelto numéricamente el sistema de ecuaciones diferenciales bajo las condiciones de los dos modelos mencionados anteriormente para las tres fases clave de la termoluminiscencia: irradiación, relajación y calentamiento. Ambos modelos reproducen con éxito la curva termoluminiscente característica para el LiF:Mg,Ti vista en los datos experimentales, pero el modelo dependiente de la temperatura consigue una descripción más precisa de las trampas poco profundas. Este modelo muestra que la trampa menos profunda (llamada trampa I, numerada del I al V en función de su profundidad) presenta una saturación más temprana, una mayor liberación térmica y un cambio en la posición e intensidad de los picos en comparación con el modelo independiente de la temperatura. Las trampas más profundas (de II a V) muestran una variación menor entre ambos modelos, lo que sugiere que su comportamiento es menos sensible a la diferencia en la expresión del factor de frecuencia.

\vspace{10pt}

Las curvas termoluminiscentes simuladas reproducen los cinco picos característicos del LiF:Mg,Ti observados en los datos experimentales. La concordancia entre la temperatura pico y los máximos de la curva termoluminiscente refuerza la validez de los modelos. Este trabajo reproduce el comportamiento conocido de la TL del LiF:Mg,Ti, y además destaca la importancia de la dependencia de la temperatura en el factor de frecuencia, ya que el modelo dependiente de la temperatura presenta una mejor capacidad para explicar la dinámica de los portadores de carga. Un trabajo futuro podría afinar aún más este modelo incorporando una dependencia de la temperatura más compleja.


\paragraph{Palabras clave:} Termoluminiscencia; LiF:Mg,Ti; Radiación; Dosimetría; Simulación









\chapter*{Abstract}
\addcontentsline{toc}{chapter}{Abstract. Keywords}

This work presents the simulation and analysis of a mathematical model for the thermoluminescent (TL) response of LiF:Mg,Ti; a material widely used in radiation dosimetry. Two models were developed to describe the physical processes underlying thermoluminescence through a set of differential equations: one with a constant frequency factor dependent on each trap and another with a temperature dependent frequency factor based on principles of quantum statistical mechanics.

\vspace{10pt}

For this Bachelor's Thesis, a series of thermoluminescence glow curves were simulated in a Python environment, numerically solving the differential equations under the conditions of both models for the three key phases of the TL process: irradiation, relaxation, and heating. Both models successfully reproduce the characteristic TL glow curve of LiF:Mg,Ti, but a more accurate representation of shallow traps is achieved with the temperature dependent model. This model shows that the shallowest trap (called trap I, as they are numbered from I to V according to their depth) exhibits earlier saturation, greater thermal release, and a shift in peak position and intensity compared to the temperature independent model. Deeper traps (II to V) showed minimal variation between models, suggesting that their behavior is less sensitive to the difference in the frequency factor's expression.

\vspace{10pt}

The simulated TL glow curves reproduced the characteristic five peaks of LiF:Mg,Ti shown in experimental data. The agreement between the peak temperature and the glow curve maxima reinforces the model's validity. This work not only reproduces the known TL behavior of LiF:Mg,Ti but also highlights the significance of temperature dependency in the frequency factor, enhancing the temperature dependent model's ability to explain charge carrier dynamics. Future work may refine this model further, potentially incorporating more complex temperature dependencies.

\paragraph{Keywords:} Thermoluminescence; LiF:Mg,Ti; Radiation; Dosimetry; Simulation
