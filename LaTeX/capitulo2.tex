\chapter{Objectives}\label{ch:2}

The aim of this work is to study the thermoluminescent behavior of LiF:Mg,Ti through a computational simulation of the physical processes involved. The dynamics of the TL process during the three key phases ---Irradiation, Relaxation and Heating--- is described by two models based on a set of differential equation one follows the general practice in the field used as reference, often referred to as the \textit{temperature independent model}, and the other dives deeper into the physics of the process and considers a further role on the temperature dependence of its parameters, referred to as the \textit{temperature dependent model}.

\vspace{10pt}

The specific objectives of this work are:
\begin{enumerate}[label=\textbf{\arabic*.}, font=\bfseries]
    \item \textbf{Develop a mathematical model} that describes the thermoluminescent behavior of LiF:Mg,Ti, based on the physical processes involved in the phenomenon. For that, it will be important to understand the fundamental mechanisms of ionic crystals, the role of impurities, and the processes of electron trapping and recombination that ultimately leads to the emission of light.
    \item \textbf{Implement and validate a numerical method} to solve the system of differential equations derived from the mathematical model, allowing for the simulation of the TL process using a set of physically accurate parameters.
    \item \textbf{Analyze and compare the results of both models} to understand the impact of temperature dependence on the TL behavior of LiF:Mg,Ti. One of the central objectives of this work is to determine how the temperature dependence affects the shape and intensity of the TL glow curves and whether both models yield similar results under certain conditions, and ultimately reproduce a similar behavior to the one observed in experimental data. From that, it will be possible to draw conclusions about the validity of the assumptions made in each model and their implications for the understanding of TL processes.
\end{enumerate}

\vspace{10pt}

Ultimately, this study aims not only to reproduce the characteristic behavior of LiF:Mg,Ti through simulation, but also to question about the physics behind the interactions of the particles involved. By comparing different approximations made into models, this work contributes to a better understanding of how different theoretical assumptions can reach a similarity with the experimental truth.