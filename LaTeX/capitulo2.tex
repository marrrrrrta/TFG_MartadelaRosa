\chapter{Objectives}\label{ch:2}

The aim of this work is to study the thermoluminescent (TL) behavior of LiF:Mg,Ti through a computational simulation of the physical processes involved. Based on the mathematical model presented in Chapter \ref{ch:3}, the objective is to numerically solve the system of differential equations that describe the dynamics of the TL process during the three key phases: Irradiation, Relaxation and Heating. For that, two mathematical models have been developed, one that follows the general practice in the field used as reference, often referred to as the \textit{temperature independent model}, and one that dives deeper into the physics of the process and considers a further role on the temperature dependence of its parameters, referred to as the \textit{temperature dependent model}.

\vspace{10pt}

The specific objectives of this work are:
\begin{enumerate}[label=\textbf{\arabic*.}, font=\bfseries]
    \item \textbf{Develop a mathematical model} that describes the thermoluminescent behavior of LiF:Mg,Ti, based on the physical processes involved in the TL phenomenon. For that, it will be important to understand the fundamental mechanisms of ionic crystals, the role of impurities, and the processes of electron trapping and recombination that ultimately leads to the emission of light.
    \item \textbf{Implement and validate a numerical method} to solve the system of differential equations derived from the mathematical model, allowing for the simulation of the TL process. Using a set of physically accurate parameters, the models should be able to reproduce the characteristic TL glow curves observed in experimental data, and be capable of tracing back the evolution of the system to its initial state.
    \item \textbf{Analyze and compare the results of both models} to understand the impact of temperature dependence on the TL behavior of LiF:Mg,Ti. One of the central objectives of this work is to determine how the temperature dependence affects the shape and intensity of the TL glow curves and whether both models yield similar results under certain conditions, and ultimately reproduce a similar behavior to the one observed in experimental data.
    \item \textbf{Interpret the simulated thermoluminescent signal} to draw conclusions about the physical processes involved in the TL phenomenon. The final goal is to obtain the TL glow curve from the simulated heating phase and extract from it key information such as the activation and peak temperatures for each trap. This will allow for a deeper understanding of the trapping and recombination processes in LiF:Mg,Ti, and how they are influenced by temperature.
\end{enumerate}

\vspace{10pt}

Ultimately, this study aims not only to reproduce the characteristic behavior of LiF:Mg,Ti through simulation, but also to question about the physics behind the interactions of the particles involved. By comparing different approximations made into models, this work contributes to a better understanding of how different theoretical assumptions can reach a similarity with the experimental truth.