\chapter*{Conclusiones}
\addcontentsline{toc}{chapter}{Conclusiones}

En este trabajo se ha desarrollado, simulado y analizado un modelo matemático para la respuesta termoluminiscente (TL) del material LiF:Mg,Ti, de acuerdo con los objetivos planteados en el Capítulo \ref{ch:2}.

\vspace{10pt}

Primero, los procesos físicos detrás de la termoluminiscncia se tradujeron en un sistema de ecuaciones diferenciales capaz de describir el comportamiento del material siguiendo una serie de aproximaciones. A partir de esta base se desarollaron dos modelos diferentes, que se bifurcan en función de la expresión del factor de frecuencia; un primer modelo, el más simple y usado como referencia, asume un valor constante específico para cada trampa (ecuación \ref{eq:freqfactor}), y un segundo modelo que incorpora una dependencia de la temperatura en el factor de frecuencia basada en la mecánica estadística cuántica (ecuación \ref{eq:freqfactor_lnT}).

\vspace{10pt}

Ambos modelos han sido implementados en un entorno de simulación numérica que es capaz de solucionar el sistema de ecuaciones diferenciales para las tres fases claves del proceso termoluminiscente: irradiación, relajación y calentamiento. Las simulaciones han conseguido imitar el proceso de llenado de las trampas durante la irradiación (Figura \ref{fig:irradiation_nievolution}), la consecuente liberación de electrones en la fase de relajación (Figura \ref{fig:relaxation_nievolution}), y la liberación térmica final de electrones durante la fase de calentamiento (Figura \ref{fig:GC_ActivationAndPeakTemperatures}). La curva de brillo termoluminiscente obtenida para ambos modelos es consistente con la forma característica del comportamiento experimental del LiF:Mg,Ti (Figura \ref{fig:ExperimentalGlowCurve}). Los resultados han sido validados mediante un gráfico de prueba (Figuras \ref{fig:irradiation_chneutrality}, \ref{fig:relaxation_chneutrality} y \ref{fig:heating_chneutrality}) realizado para cada fase, que asegura que el sistema se mantiene eléctricamente neutro.

\vspace{10pt}

Tras comparar los resultados de ambos modelos, se ha demostrado que el modelo dependiente de la temperatura es capaz de describir mejor el comportamiento de la trampa I, ya que muestra una saturación más rápida durante la irradiación, una disminución más pronunciada en la densidad de electrones durante la relajación y una recombinación más temprana en la fase de calentamiento. También se observa un ligero desplazamiento en la posición e intensidad de los picos en la curva de brillo termoluminiscente (Figura \ref{fig:heating_TLGlowCurve}). Las trampas más profundas, de II a V, exhiben un comportamiento similar en ambos modelos, lo que indica que sus energías de activación, al ser más altas, hacen que sean menos sensibles a la dependencia de temperatura del factor de frecuencia.

\vspace{10pt}

Las curvas de brillo termoluminiscente simuladas (Figura \ref{fig:heating_TLGlowCurve}) son bastante similares a los datos experimentales. El gráfico muestra sus cinco picos característicos, cada uno coincidiendo con las temperaturas de activación y pico identificadas en los gráficos de evolución de $n_i(t)$. Esta estrecha correspondencia entre la temperatura de máxima liberación de cada trampa en la Figura \ref{fig:GC_ActivationAndPeakTemperatures} y su pico luminescente en la Figura \ref{fig:heating_TLGlowCurve}, confirma que los picos de intensidad reflejan de manera fiable las energías de activación de las trampas y sus probabilidades de recombinación.

\vspace{10pt}

En resumen, este estudio no solo reproduce el comportamiento termoluminiscente visto experimentalmente en el LiF:Mg,Ti mediante simulación por ordenador, sino que también demuestra que la introducción de una dependencia de la temperatura en el factor de frecuencia puede mejorar la capacidad del modelo para explicar la dinámica de los portadores de carga involucrados en el proceso. Estos hallazgos subrayan la importancia del sutil efecto de la temperatura en el modelado termoluminiscente, y abren camino para futuros trabajos que refinen aún más el modelo, potencialmente incorporando una dependencia de temperatura más compleja. El contenido de este Trabajo Fin de Grado se presentará en la 51ª Reunión Anual de la Sociedad Nuclear Española el 24 de septiembre de 2025.


\chapter*{Conclusions}
\addcontentsline{toc}{chapter}{Conclusions}

In this work, a mathematical model for the thermoluminescent (TL) response of LiF:Mg,Ti was developed, simulated and analyzed in accordance with the objectives set out in Chapter \ref{ch:2}. 

\vspace{10pt}

First, the underlying physical processes of thermoluminescence were translated into a set of differential equations capable of describing the behavior of the material with a series of approximations. Two different models branched out over the temperature dependency in the electron's frequency factor (equation \ref{eq:freqfactor}); the first and most simple one assumed a constant value and was used as a reference, and the second one incorporated a temperature dependency based on quantum statistical mechanics (equation \ref{eq:freqfactor_lnT}). 

\vspace{10pt}

Both models were then implemented in a numerical simulation environment that was able to solve these equations across all three key phases of the TL process: irradiation, relaxation and heating. The simulations successfully reproduced the processes of filling of traps during irradiation (Figure \ref{fig:irradiation_nievolution}), the subsequent liberation of electrons in the relaxation phase (Figure \ref{fig:relaxation_nievolution}), and the final thermal release of electrons during the heating phase (Figure \ref{fig:GC_ActivationAndPeakTemperatures}). The TL glow curves obtained for both models were consistent with the characteristic shape of the experimental behavior of LiF:Mg,Ti (Figure \ref{fig:ExperimentalGlowCurve}). The results were validated through a test graph (Figures \ref{fig:irradiation_chneutrality}, \ref{fig:relaxation_chneutrality}, and \ref{fig:heating_chneutrality}) done for every phase, that assured that the system stayed electrically neutral.

\vspace{10pt}

A comparison between the two models was made, and revealed that the temperature dependent model was able to better reproduce the behavior of trap I, showing earlier saturation during irradiation, a steeper decrease in occupancy during relaxation and an early recombination in the heating phase. It also showed a slight shift in peak position and intensity in the glow curve (Figure \ref{fig:heating_TLGlowCurve}). In contrast, deeper traps from II to V exhibited similar behavior in both models, indicating that their higher activation energies rendered them less sensitive to the frequency factor's temperature dependency.

\vspace{10pt}

The simulated TL glow curves (Figure \ref{fig:heating_TLGlowCurve}) display a good agreement with the experimental data. The plot displays its five distinct peaks, each matching the activation and peak temperatures identified from the $n_i(t)$ evolution plots. The close correspondence between each trap's maximum release rate temperature (Figure \ref{fig:GC_ActivationAndPeakTemperatures}) and its luminescent peak (Figure \ref{fig:heating_TLGlowCurve}) confirms that the glow peaks reliably reflect the trap activation energies and recombination probabilities.

\vspace{10pt}

In summary, this study not only reproduces the known TL behavior of LiF:Mg,Ti via computer simulation, but also demonstrates how introducing a temperature dependency in the frequency factor can enhance the model's ability to explain the dynamics of the charge carriers involved in the process. These findings underscore the importance of the subtle effect of temperature in the TL modeling, and pave way for future work to refine the model further, potentially incorporating a more complex temperature dependency. The contents of this final thesis will be presented in the 51$^{\text{st}}$ Annual Meeting of the Spanish Nuclear Society on the 24$^{\text{th}}$ of September 2025.