\chapter{Introduction}


%\section{Introduction}
% Por qué tú
% Por qué importa
% Qué deben saber

As a bystander, one may pass life without thinking of the things that surrounds us. One may have had the misfortune of entering on an MRI, or the responsability to carry a dosimeter in a nuclear plant, and stepped out the room as it is. Life can go on unquestioned, and one may get out of that PET scan without thinking of the source of that awful noise. 

\vspace{10pt}

There's beauty in the mundane, and the world is full of wonders. The universe is a complex system of interactions, and we know a very small part of it. We are surrounded by radiation, and we are constantly exposed to it. It is a natural phenomenon that has been present since the beginning of time, and it is an integral part of our existence. There are answers for those who wonder, and this work is a very small step towards it.

\vspace{10pt}

Luminescence is a phenomenon familiar to us; and goes through our lives like a commercial break. We see it in the glow of a firefly, the sparkle of a diamond, or the light emitted by a fluorescent lamp. In a nutshell, it is a process where energy is absorbed and re-emitted as light, leaving a trail behind, and can be triggered by various stimuli, like heat, light or radiation. This broad notion is the reason why the study of luminescence has practical uses in many fields. One of those fields of use is the detection of ionizing radiation, a field generally known as ``dosimetry''. The amount of radiation absorbed by a material can be measured by the amount of light emitted when the material is stimulated, and this is the basis for many dosimetry techniques. These luminescence-based methods for detecting ionizing radiation have played a central role in radiation research since the earliest discoveries of radiation, as they exploit the ability of specific materials to emit light when exposed to ionizing radiation, to detect and quantify the radiation received, further expanding the knowledge its efect. 

\vspace{10pt}

One of the most commonly used materials in this context is lithium fluoride doped with magnesium and titanium (LiF:Mg,Ti). This material exhibits thermoluminescence, a phenomenon where trapped electrons in the crystal lattice are released upon heating, emitting light in the process. The study of such materials is crucial for improving the accuracy and efficiency of radiation detection systems, and it will be the cornerstone of this work. 

% Why ????????
