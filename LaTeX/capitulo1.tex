\chapter{Introduction}


%\section{Introduction}

As a bystander, one may pass life without thinking of the things that surrounds us. One may have had the misfortune of entering on an MRI, or the responsability to carry a dosimeter in a nuclear plant, and stepped out the room as it is. Life can go on unquestioned, and one may step out of that PET scan without thinking of the source of that awful noise. 

Luminescence-based methods for detecting ionizing radiation have played a central role in radiation research since the earliest discoveries of radiation, radioactivity and atomic structure. 
These techniques exploit the ability of specific materials to emit light when exposed to ionizing radiation. This emitted light intensity is proportional to the radiation dose, making it possible to measure and to quantify the exposure.

One of the most commonly used materials in this context is lithium fluoride doped with magnesium and titanium (LiF:Mg,Ti). This material exhibits thermoluminescence, a phenomenon where trapped electrons in the crystal lattice are released upon heating, emitting light in the process. The study of such materials is crucial for improving the accuracy and efficiency of radiation detection systems, and it will be the cornerstone of this work. 

