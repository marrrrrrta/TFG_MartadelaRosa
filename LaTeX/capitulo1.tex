\chapter{Introduction}


%\section{Introduction}
% Por qué tú
% Por qué importa
% Qué deben saber

As a bystander, one may pass life without thinking of the things that surrounds us. One may have had the misfortune of entering on an MRI, or the responsability to carry a dosimeter in a nuclear plant, and stepped out the room as it is. Life can go on unquestioned, and one may get out of that PET scan without thinking of the source of that awful noise. 

\vspace{10pt}

There's beauty in the mundane, and the world is full of wonders. The universe is a complex system of interactions, and we know a very small part of it. We are surrounded by radiation, and we are constantly exposed to it. It is a natural phenomenon that has been present since the beginning of time, and it is an integral part of our existence. There are answers for those who wonder, and this work is a very small step towards it.

\vspace{10pt}

Luminescence is a phenomenon familiar to us; and goes through our lives like a commercial break. We see it in the glow of a firefly, the sparkle of a diamond, or the light emitted by a fluorescent lamp. In a nutshell, it is a process where energy is absorbed and re-emitted as light, leaving a trail behind, and can be triggered by various stimuli, like heat, light or radiation. This broad notion is the reason why the study of luminescence has practical uses in many fields. One of those fields of use is the detection of ionizing radiation, a field generally known as ``dosimetry''. The amount of radiation absorbed by a material can be measured by the amount of light emitted when the material is stimulated, and this is the basis for many dosimetry techniques. These luminescence-based methods for detecting ionizing radiation have played a central role in radiation research since the earliest discoveries of radiation, as they exploit the ability of specific materials to emit light when exposed to ionizing radiation, to detect and quantify the radiation received, further expanding the knowledge its efect. 

\vspace{10pt}

One of the most commonly used materials in this context is lithium fluoride doped with magnesium and titanium (LiF:Mg,Ti). This material exhibits thermoluminescence, a phenomenon where after irradiation, its internal structure captures a memory of the event, in the form of trapped electrons. Upon heating, these trapped charges are released; recombining and emitting photons in the process. The resulting light ---that we know to be called luminescence---, if recorded as a function of temperature, produces what is called a thermoluminescence (TL) glow curve. % fix

But knowing that LiF:Mg, Ti emits light when heated is only the beginning. The real challenge lies in understanding it. Interpreting it. The glow curve, with its peaks and valleys, is more than a passive result. It is a message from within the material, shaped by the dance of the electrons accross the imperfections of the lattice, and it tells a story ---if we know how to read it.

\vspace{10pt}

To make sense of that story, one must model it. That is, to simulate the physical processes that give rise to the observed glow, and to see whether our mathematical model truly mirror nature. Can we, with a set of parameters and approximations, recreate the fingerprint of radiation? Can we extract from that curve a clear image of the processes within?

\vspace{10pt}

This is where this work begins. At the heart of this discourse lies the attempt to reproduce the TL glow curve of LiF:Mg, Ti through computational modeling. This effort leans on the shoulders of many brilliant scientists with an insatiable hunger for one of the many misteries of the universe, one where electrons takes us across energy barriers, where recombination is probabilistic, and where heat becomes the catalyst of understanding.

\vspace{10pt}

But even the best models are incomplete. Many assume constant parameters ---fixed frequency factors and activation energies, unchanged by temperature or entropy. Reality, as often, lies in a space in between. In these pages, it will be investigated what happens when we let these parameters breathe. By introducing temperature-dependent frequency factors, we will investigate how this modification reshapes the predicted glow curve. Is the model improved? Do we get closer to the experimental results? Can it go beyond known data and predict a hypothetical future case?


\vspace{10pt}

Ultimately, the motivation is simple: to understand. To refine our lens on thermoluminescence; to bridge the gap between theory and experience, and to contribute, even in the smallest way, to the slow unraveling of the invisible questions that shape our everyday lives.


\begin{comment}
    This curve is more than just a trace---it is a fingerprint of the radiation dose absorbed by the material, and so it provides clues to pinpoint how much radiation the material has actually been exposed to. Each peak in the curve corresponds to a distinct defect in the crystal's structure, and their properties of intensity and position in the temperature axis can be used to determine the radiation dose. By unraveling the meaning behind these peaks, we uncover the foundation for building a realiable dose-response relationship. By doing so, we gain not just a method, but a powerful lens through which we can peek at the invisible trace of radiation becomes measurable. This quiet transformation ---from light to knowledge--- is the very essence of dosimetry.
\end{comment}
% Why ????????
