\chapter{Simulations} \label{ch:4}

\section{Materials used for simulations} \label{sec:materials}

The materials used for this work are all the digital tools required for the simulation of the behaviour of our material of choice. In our case, the project was made using \textit{Python} in a Jupyter Notebook. The programs written are available on the GitHub repository called \texttt{TFG\_MartadelaRosa}. 

\vspace{10pt}

The libraries used for the simulations are:
\begin{itemize}
    \item \texttt{numpy}: This library is used for numerical calculations and array manipulations.
    \item \texttt{matplotlib}: This library is used for plotting graphs and visualizing data.
    \item \texttt{pandas}: This library is used for data manipulation and analysis, providing data structures like DataFrames.
    \item \texttt{scipy}: This library is used for scientific computing and includes functions for optimization, integration, interpolation, and more. In particular to solve the differential equations that describe the kinetics of the TL process.
\end{itemize}



\section{Simulations} \label{sec:simulations}

To compute a simulation of the TL process described in Section \ref{sec:modelo}, after defining the parameters of the model, we need to solve the differential equations that describe the kinetics of the TL process. The equations are solved using the \texttt{odeint} function from the \texttt{scipy.integrate} library, which is a powerful tool for solving ordinary differential equations. The function takes as input the system of equations, the initial conditions, and the time points at which we want to evaluate the solution. The output is an array containing the values of the variables at each time point, which is then saved to their corresponding variables, used to plot the results. They are brought to life using the \texttt{matplotlib} library, and for every resolution of the differential equations, we will present three graphs as the result.

\vspace{10pt}


